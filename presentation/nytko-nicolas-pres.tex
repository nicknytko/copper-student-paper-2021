%%
% Please see https://bitbucket.org/rivanvx/beamer/wiki/Home for obtaining beamer.
%%
\documentclass{beamer}
%\usetheme{Warsaw}

\usepackage{blkarray}
\usepackage{bigstrut}
\usepackage{amsmath}
\usepackage{bbm}
\usepackage{harpoon}
\usefonttheme[onlymath]{serif}

\makeatletter
\let\BA@quicktrue\BA@quickfalse
\makeatother

\renewcommand{\vec}[1]{ {\bf #1} }

\title{A Supervised Learning Approach to Predicting Multigrid Convergence}
\author[me]{Nicolas Nytko\\[3mm]Matthew West, Luke Olson, Scott MacLachlan}
\date{\today}

\begin{document}
\frame{\titlepage}

% Overview
\begin{frame}
  \frametitle{Overview}
\end{frame}

% Poisson intro
\begin{frame}
  \frametitle{Poisson Problem}
  \begin{itemize}
  \item Look at the 1D variable coefficients case w/ homogeneous Dirichlet conditions
    \[ -\nabla \cdot \left(k\left(\vec{x}\right) \nabla \vec{y} \right) = f \]
    \[ \Omega = \left[-1, 1\right] \quad \partial\Omega = 0 \]
  \item Discretized on $N=31$ internal points using finite differences, $k\left(\vec{x}\right)$ is discretized on midpoints to preserve symmetry.
  \item For arbitrary C/F splitting, can we predict convergence rate and optimal relaxation weight?
  \end{itemize}
\end{frame}

\end{document}
